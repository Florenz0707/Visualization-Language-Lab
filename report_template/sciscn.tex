% !TeX encoding = UTF-8
% !TeX program = xelatex
% !TeX spellcheck = en_US

%-----------------------------------------------------------------------
% 中国科学: 信息科学 中文模板, 请用 CCT-LaTeX 编译, 也可以在Overleaf中使用XeLaTeX直接编译
% http://scis.scichina.com
% 致谢: 本模版的制作得到下列老师的热心支持和帮助:
% 南开大学程明明老师: 帮助制作了在线投稿模板的cls文件;
% 天津大学郝建业老师: 帮助制作了bib格式文件scis.bst;
% 重庆邮电大学曾宪华老师: 帮助修改了cls文件中的部分代码;
% 在此特别表示感谢!
%-----------------------------------------------------------------------
% Overleaf在线编译方法:
% (1) 打开https://www.overleaf.com/project/66beaf29c51c8d8931397a15
% (2) 点左上角"Menu"-->"Copy Project", 把本项目拷贝一份到自己的账号下
% (3) 在这个新项目里编辑自己的论文, 目前默认就是使用XeLaTex编译, 如果编译不成功, 请在"Menu"中检查编译器是否是XeLaTex
% (4) 完成后, 可视化pdf检查无误, 点右边顶部"Recompile"旁边的下载按钮, 下载pdf即可
%-----------------------------------------------------------------------

\documentclass{SCIS2024cn}
%%%%%%%%%%%%%%%%%%%%%%%%%%%%%%%%%%%%%%%%%%%%%%%%%%%%%%%
%%% 作者附加的定义
%%% 常用环境已经加载好, 不需要重复加载
%%%%%%%%%%%%%%%%%%%%%%%%%%%%%%%%%%%%%%%%%%%%%%%%%%%%%%%


%%%%%%%%%%%%%%%%%%%%%%%%%%%%%%%%%%%%%%%%%%%%%%%%%%%%%%%
%%% 开始
%%%%%%%%%%%%%%%%%%%%%%%%%%%%%%%%%%%%%%%%%%%%%%%%%%%%%%%
\begin{document}

%%%%%%%%%%%%%%%%%%%%%%%%%%%%%%%%%%%%%%%%%%%%%%%%%%%%%%%
%%% 作者不需要修改此处信息
\ArticleType{论文}
%\SpecialTopic{}
%\Luntan{中国科学院学部\quad 科学与技术前沿论坛}
\Year{2024}
\Vol{54}
\No{9}
\BeginPage{1}
\DOI{}
\ReceiveDate{2024--00--00}
\ReviseDate{2024--00--00}
\AcceptDate{2024--00--00}
\OnlineDate{}
\AuthorMark{作者1等}
\AuthorCitation{作者1, 作者2, 作者3, 等}
%\enAuthorCitation{Xing M1, Xing M2, Xing M3, et al}
%%%%%%%%%%%%%%%%%%%%%%%%%%%%%%%%%%%%%%%%%%%%%%%%%%%%%%%

\title{正文标题}{引用的标题}

%\entitle{Title}{Title for citation}

\author[1]{组长}{Ming XING1}{}
\author[2]{队员1}{Ming XING2}{{xingming2@xxxx.xxx}}
\author[3]{队员2}{Ming XING3}{}
%\author[3]{作者4}{Ming XING4}{}
%\author[1,2]{作者5}{Ming XING5}{}
%\author[1,2]{作者6}{\\Ming XING6}{{xingming6@xxxx.xxx}}
%\author[1,2]{作者7}{Ming XING7}{}

%若英文部分的emaillist太长需要换行的话,形式单独写在这里
%\enauthoremaillist{xingming1@xxxx.xxx, xingming2@xxxx.xxx, xingming3@xxxx.xxx, xingming4@xxxx.xxx, xingming5@xxxx.xxx}

%\comment{\dag~同等贡献}
%\encomment{\dag~Equal contribution}

\address[1]{计算机科学与技术专业}{Affiliation, City {\rm 000000}, Country}
\address[2]{软件工程专业}{Affiliation, City {\rm 000000}, Country}
\address[3]{人工智能专业}{Affiliation, City {\rm 000000}, Country}

%\Foundation{可视语言与信息可视化(双语) 综合项目报告}

\abstract{摘要主要包括本文的研究目的、方法、结果和结论, 注意突出创新点. 应避免出现图、表、公式、参考文献引用等. 对应的英文摘要长度在200词左右.}

%\enabstract{An abstract (about 200 words) is a summary of the content of the manuscript. It should briefly describe the research purpose, method, result and conclusion. The extremely professional terms, special signals, figures, tables, chemical structural formula, and equations should be avoided here, and citation of references is not allowed.}

\keywords{关键词1, 关键词2, 关键词3, 关键词4, 关键词5}
%\enkeywords{keyword1, keyword2, keyword3, keyword4, keyword5}

\maketitle

\section{一级标题}

正文开始. 直接提到文献~\cite{test1,test2,test3,test5,test4,test6}中, 使用平排, 引用他人工作使用小上标~\upcite{test7,test8,test9,test10,test11,test12}.

使用英文逗号``,''、句号``.''、冒号``:''、分号``;''.

\begin{definition}[定义名,可省略]\label{def1}
这是一个定义.
\end{definition}


\section{一些常用的格式}
\subsection{图片}
图片如\ref{fig1}所示.
\begin{figure}[!t]
\centering
\includegraphics{fig/fig1.eps}
\cnenfigcaption{(网络版彩图) 图题}{(Color online) Caption}
\label{fig1}
\end{figure}


\subsection{公式}
%%% 公式组
\begin{eqnarray}
\nonumber
X&=&[x_{11},x_{12},\ldots,x_{ij},\ldots ,x_{n-1,n}]^{\rm T},\\
\nonumber
\varepsilon&=&[e_{11},e_{12},\ldots ,e_{ij},\ldots ,e_{n-1,n}],\\
\nonumber
T&=&[t_{11},t_{12},\ldots ,t_{ij},\ldots ,t_{n-1,n}].
\end{eqnarray}

%%% 条件公式
\begin{eqnarray}
\sum_{j=1}^{n}x_{ij}-\sum_{k=1}^{n}x_{ki}=
\left\{
\begin{aligned}
1,&\quad i=1,\\
0,&\quad i=2,\ldots ,n-1,\\
-1,&\quad i=n.
\end{aligned}
\right.
\label{eq1}
\end{eqnarray}

\subsection{表格}
表格如表\ref{tab1}所示.
\begin{table}[!t]
\cnentablecaption{表题}{Caption}
\label{tab1}
\footnotesize
\tabcolsep 40pt %space between two columns. 用于调整列间距
\begin{tabular*}{\textwidth}{cccc}
\toprule
  Title a & Title b & Title c & Title d \\\hline
  Aaa & Bbb & Ccc & Ddd\\
  Aaa & Bbb & Ccc & Ddd\\
  Aaa & Bbb & Ccc & Ddd\\
\bottomrule
\end{tabular*}
\end{table}

\subsection{算法}
算法如算法\ref{alg1}所示.
\begin{algorithm}
%\floatname{algorithm}{Algorithm}%更改算法前缀名称
\renewcommand{\algorithmicrequire}{\heiti{\textbf 输入:}}% 更改输入名称
\renewcommand{\algorithmicensure}{\heiti{\textbf 主迭代:}}% 更改输出名称
\newcommand{\LASTCON}{\item[\algorithmiclastcon]}
\newcommand{\algorithmiclastcon}{\heiti{\textbf 输出:}}% 更改输出名称
\footnotesize
\caption{算法标题}
\label{alg1}
\begin{algorithmic}[1]
    \REQUIRE $n \geq 0 \vee x \neq 0$;
    \ENSURE $y = x^n$;
    \STATE $y \Leftarrow 1$;
    \IF{$n < 0$}
        \STATE $X \Leftarrow 1 / x$;
        \STATE $N \Leftarrow -n$;
    \ELSE
        \STATE $X \Leftarrow x$;
        \STATE $N \Leftarrow n$;
    \ENDIF
    \WHILE{$N \neq 0$}
        \IF{$N$ is even}
            \STATE $X \Leftarrow X \times X$;
            \STATE $N \Leftarrow N / 2$;
        \ELSE[$N$ is odd]
            \STATE $y \Leftarrow y \times X$;
            \STATE $N \Leftarrow N - 1$;
        \ENDIF
    \ENDWHILE
    \LASTCON
\end{algorithmic}
\end{algorithm}

%%%%%%%%%%%%%%%%%%%%%%%%%%%%%%%%%%%%%%%%%%%%%%%%%%%%%%%
%%% 致谢
%%% 非必选
%%%%%%%%%%%%%%%%%%%%%%%%%%%%%%%%%%%%%%%%%%%%%%%%%%%%%%%
%\Acknowledgements{致谢.}

%%%%%%%%%%%%%%%%%%%%%%%%%%%%%%%%%%%%%%%%%%%%%%%%%%%%%%%
%%% 补充材料说明
%%% 非必选
%%%%%%%%%%%%%%%%%%%%%%%%%%%%%%%%%%%%%%%%%%%%%%%%%%%%%%%
%\Supplements{补充材料.}

%%%%%%%%%%%%%%%%%%%%%%%%%%%%%%%%%%%%%%%%%%%%%%%%%%%%%%%
%%% 参考文献, {}为引用的标签, 数字/字母均可, ~使
% 文中上标引用: ~\upcite{1,2}
% 文中正常引用: ~\cite{1,2}
%%% 可使用bib文件, 例如:
% \bibliographystyle{scis}
% \bibliography{ref}
%%% 也可使用\bibitem{}, 例如:
% \begin{thebibliography}[99]
% \bibitem{test1} Turing A M. Computing Machinery and Intelligence. Springer, 2009
% \end{thebibliography}
%%%%%%%%%%%%%%%%%%%%%%%%%%%%%%%%%%%%%%%%%%%%%%%%%%%%%%%
% \bibliographystyle{scis}
% \bibliography{ref}

\begin{thebibliography}{99}
\bibitem{test1} Turing A M. Computing Machinery and Intelligence. Springer, 2009
\bibitem{test2} Kalai A, Vempala S. Efficient algorithms for online decision problems. J Comput Syst Sci, 2005, 71: 291–307
\bibitem{test3} Huang K Q, Xing J L, Zhang J G, et al. Intelligent technologies of human-computer gaming. Sci Sin Inform, 2020, 50:
540–550 [黄凯奇, 兴军亮, 张俊格, 等. 人机对抗智能技术. 中国科学: 信息科学, 2020, 50: 540–550]
\bibitem{test4} Yang Y, Hao J, Liao B, et al. Qatten: a general framework for cooperative multiagent reinforcement learning. arXiv, 2020.
arXiv:1611.01144
\bibitem{test5} Smith L, Gasser M. The development of embodied cognition: six lessons from babies. Artif Life, 2005, 11: 13–29
\bibitem{test6} Duan J, Yu S, Tan H L, et al. A survey of embodied AI: from simulators to research tasks. IEEE Trans Emerg Topic Comput
Intell, 2022, 6: 230–244
\bibitem{test7} Rashid T, Samvelyan M, De Witt C S, et al. Monotonic value function factorisation for deep multi-agent reinforcement
learning. J Mach Learn Res, 2020, 21: 7234–7284
\bibitem{test8} Turing A M. Computing machinery and intelligence. Mind, 1950, 59: 433
\bibitem{test9} Foerster J, Farquhar G, Afouras T, et al. Counterfactual multi-agent policy gradients. In: Proceedings of the 32nd AAAI
Conference on Artificial Intelligence, 2018. 2974–2982
\bibitem{test10} Sunehag P, Lever G, Gruslys A, et al. Value-decomposition networks for cooperative multi-agent learning based on team
reward. In: Proceedings of the 17th International Conference on Autonomous Agents and Multiagent Systems, 2018. 2085–
2087
\bibitem{test11} Freund Y, Schapire R E. Game theory, on-line prediction and boosting. In: Proceedings of the 9th Annual Conference on
Computational Learning Theory, 1996. 325–332
\bibitem{test12} Devlin J, Chang M, Lee K, et al. Bert: pre-training of deep bidirectional transformers for language understanding. In:
Conference of the North American Chapter of the Association for Computational Linguistics: Human Language Technologies,
NAACL-HLT, 2019. 4171–4186
\end{thebibliography}
%%%%%%%%%%%%%%%%%%%%%%%%%%%%%%%%%%%%%%%%%%%%%%%%%%%%%%%
%%% 附录章节, 自动从A编号, 以\section开始一节
%%% 非必选
%%%%%%%%%%%%%%%%%%%%%%%%%%%%%%%%%%%%%%%%%%%%%%%%%%%%%%%
%\begin{appendix}
%\section{附录}
%附录从这里开始.
%\begin{figure}[H]
%\centering
%%\includegraphics{fig1.eps}
%\cnenfigcaption{附录里的图}{Caption}
%\label{fig1}
%\end{figure}
%\end{appendix}


%%%%%%%%%%%%%%%%%%%%%%%%%%%%%%%%%%%%%%%%%%%%%%%%%%%%%%%
%%% 自动生成英文标题部分
%%%%%%%%%%%%%%%%%%%%%%%%%%%%%%%%%%%%%%%%%%%%%%%%%%%%%%%
%\newpage
%\makeentitle

%%%%%%%%%%%%%%%%%%%%%%%%%%%%%%%%%%%%%%%%%%%%%%%%%%%%%%%
%%% 补充材料, 以附件形式作网络在线, 不出现在印刷版中
%%% 不做加工和排版, 仅用于获得图片和表格编号
%%% 自动从I编号, 以\section开始一节
%%% 可以没有\section
%%%%%%%%%%%%%%%%%%%%%%%%%%%%%%%%%%%%%%%%%%%%%%%%%%%%%%%
%\begin{supplement}
%\section{supplement1}
%自动从I编号, 以section开始一节.
%\begin{figure}[H]
%\centering
%\includegraphics{fig1.eps}
%\cnenfigcaption{补充材料里的图}{Caption}
%\label{fig1}
%\end{figure}
%\end{supplement}

\end{document}


%%%%%%%%%%%%%%%%%%%%%%%%%%%%%%%%%%%%%%%%%%%%%%%%%%%%%%%
%%% 本模板使用的latex排版示例
%%%%%%%%%%%%%%%%%%%%%%%%%%%%%%%%%%%%%%%%%%%%%%%%%%%%%%%

%%% 章节
\section{}
\subsection{}
\subsubsection{}


%%% 普通列表
\begin{itemize}
\item Aaa aaa.
\item Bbb bbb.
\item Ccc ccc.
\end{itemize}

%%% 自由编号列表
\begin{itemize}
\itemindent 4em
\item[(1)] Aaa aaa.
\item[(2)] Bbb bbb.
\item[(3)] Ccc ccc.
\end{itemize}

%%% 定义、定理、引理、推论等, 可用下列标签
%%% definition 定义
%%% theorem 定理
%%% lemma 引理
%%% corollary 推论
%%% axiom 公理
%%% propsition 命题
%%% example 例
%%% exercise 习题
%%% solution 解名
%%% notation 注
%%% assumption 假设
%%% remark 注释
%%% property 性质
%%% []中的名称可以省略, \label{引用名}可在正文中引用
\begin{definition}[定义名]\label{def1}
定义内容.
\end{definition}



%%% 单图
%%% 可在文中使用图\ref{fig1}引用图编号
\begin{figure}[!t]
\centering
\includegraphics{fig1.eps}
\cnenfigcaption{中文图题}{Caption}
\label{fig1}
\end{figure}

%%% 并排图
%%% 可在文中使用图\ref{fig1}、图\ref{fig2}引用图编号
\begin{figure}[!t]
\centering
\begin{minipage}[c]{0.48\textwidth}
\centering
\includegraphics{fig1.eps}
\end{minipage}
\hspace{0.02\textwidth}
\begin{minipage}[c]{0.48\textwidth}
\centering
\includegraphics{fig2.eps}
\end{minipage}\\[3mm]
\begin{minipage}[t]{0.48\textwidth}
\centering
\cnenfigcaption{中文图题1}{Caption1}
\label{fig1}
\end{minipage}
\hspace{0.02\textwidth}
\begin{minipage}[t]{0.48\textwidth}
\centering
\cnenfigcaption{中文图题2}{Caption2}
\label{fig2}
\end{minipage}
\end{figure}

%%% 并排子图
%%% 需要英文分图题 (a)...; (b)...
\begin{figure}[!t]
\centering
\begin{minipage}[c]{0.48\textwidth}
\centering
\includegraphics{subfig1.eps}
\end{minipage}
\hspace{0.02\textwidth}
\begin{minipage}[c]{0.48\textwidth}
\centering
\includegraphics{subfig2.eps}
\end{minipage}
\cnenfigcaption{中文图题}{Caption}
\label{fig1}
\end{figure}

%%% 算法
%%% 可在文中使用 算法\ref{alg1} 引用算法编号
\begin{algorithm}
%\floatname{algorithm}{Algorithm}%更改算法前缀名称
%\renewcommand{\algorithmicrequire}{\textbf{Input:}}% 更改输入名称
%\renewcommand{\algorithmicensure}{\textbf{Output:}}% 更改输出名称
\footnotesize
\caption{算法标题}
\label{alg1}
\begin{algorithmic}[1]
    \REQUIRE $n \geq 0 \vee x \neq 0$;
    \ENSURE $y = x^n$;
    \STATE $y \Leftarrow 1$;
    \IF{$n < 0$}
        \STATE $X \Leftarrow 1 / x$;
        \STATE $N \Leftarrow -n$;
    \ELSE
        \STATE $X \Leftarrow x$;
        \STATE $N \Leftarrow n$;
    \ENDIF
    \WHILE{$N \neq 0$}
        \IF{$N$ is even}
            \STATE $X \Leftarrow X \times X$;
            \STATE $N \Leftarrow N / 2$;
        \ELSE[$N$ is odd]
            \STATE $y \Leftarrow y \times X$;
            \STATE $N \Leftarrow N - 1$;
        \ENDIF
    \ENDWHILE
\end{algorithmic}
\end{algorithm}

%%% 简单表格
%%% 可在文中使用 表\ref{tab1} 引用表编号
\begin{table}[!t]
\cnentablecaption{表题}{Caption}
\label{tab1}
\footnotesize
\tabcolsep 49pt %space between two columns. 用于调整列间距
\begin{tabular*}{\textwidth}{cccc}
\toprule
  Title a & Title b & Title c & Title d \\\hline
  Aaa & Bbb & Ccc & Ddd\\
  Aaa & Bbb & Ccc & Ddd\\
  Aaa & Bbb & Ccc & Ddd\\
\bottomrule
\end{tabular*}
\end{table}

%%% 换行表格
\begin{table}[!t]
\cnentablecaption{表题}{Caption}
\label{tab1}
\footnotesize
\def\tabblank{\hspace*{10mm}} %blank leaving of both side of the table. 左右两边的留白
\begin{tabularx}{\textwidth} %using p{?mm} to define the width of a column. 用p{?mm}控制列宽
{@{\tabblank}@{\extracolsep{\fill}}cccp{100mm}@{\tabblank}}
\toprule
  Title a & Title b & Title c & Title d \\\hline
  Aaa & Bbb & Ccc & Ddd ddd ddd ddd.

  Ddd ddd ddd ddd ddd ddd ddd ddd ddd ddd ddd ddd ddd ddd ddd ddd ddd ddd ddd ddd ddd ddd ddd ddd ddd ddd ddd ddd ddd ddd ddd.\\
  Aaa & Bbb & Ccc & Ddd ddd ddd ddd.\\
  Aaa & Bbb & Ccc & Ddd ddd ddd ddd.\\
\bottomrule
\end{tabularx}
\end{table}

%%% 单行公式
%%% 可在文中使用 (\ref{eq1})式 引用公式编号
%%% 如果是句子开头, 使用 公式(\ref{eq1}) 引用
\begin{equation}
A(d,f)=d^{l}a^{d}(f),
\label{eq1}
\end{equation}

%%% 不编号的单行公式
\begin{equation}
\nonumber
A(d,f)=d^{l}a^{d}(f),
\end{equation}

%%% 公式组
\begin{eqnarray}
\nonumber
&X=[x_{11},x_{12},\ldots,x_{ij},\ldots ,x_{n-1,n}]^{\rm T},\\
\nonumber
&\varepsilon=[e_{11},e_{12},\ldots ,e_{ij},\ldots ,e_{n-1,n}],\\
\nonumber
&T=[t_{11},t_{12},\ldots ,t_{ij},\ldots ,t_{n-1,n}].
\end{eqnarray}

%%% 条件公式
\begin{eqnarray}
\sum_{j=1}^{n}x_{ij}-\sum_{k=1}^{n}x_{ki}=
\left\{
\begin{aligned}
1,&\quad i=1,\\
0,&\quad i=2,\ldots ,n-1,\\
-1,&\quad i=n.
\end{aligned}
\right.
\label{eq1}
\end{eqnarray}

%%% 其他格式
\footnote{Comments.} %footnote. 脚注
\raisebox{-1pt}[0mm][0mm]{xxxx} %put xxxx upper or lower. 控制xxxx的垂直位置

%%% 图说撑满
\Caption\protect\linebreak \leftline{Caption}
